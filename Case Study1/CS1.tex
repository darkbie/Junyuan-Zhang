% Options for packages loaded elsewhere
\PassOptionsToPackage{unicode}{hyperref}
\PassOptionsToPackage{hyphens}{url}
%
\documentclass[
]{article}
\usepackage{lmodern}
\usepackage{amssymb,amsmath}
\usepackage{ifxetex,ifluatex}
\ifnum 0\ifxetex 1\fi\ifluatex 1\fi=0 % if pdftex
  \usepackage[T1]{fontenc}
  \usepackage[utf8]{inputenc}
  \usepackage{textcomp} % provide euro and other symbols
\else % if luatex or xetex
  \usepackage{unicode-math}
  \defaultfontfeatures{Scale=MatchLowercase}
  \defaultfontfeatures[\rmfamily]{Ligatures=TeX,Scale=1}
\fi
% Use upquote if available, for straight quotes in verbatim environments
\IfFileExists{upquote.sty}{\usepackage{upquote}}{}
\IfFileExists{microtype.sty}{% use microtype if available
  \usepackage[]{microtype}
  \UseMicrotypeSet[protrusion]{basicmath} % disable protrusion for tt fonts
}{}
\makeatletter
\@ifundefined{KOMAClassName}{% if non-KOMA class
  \IfFileExists{parskip.sty}{%
    \usepackage{parskip}
  }{% else
    \setlength{\parindent}{0pt}
    \setlength{\parskip}{6pt plus 2pt minus 1pt}}
}{% if KOMA class
  \KOMAoptions{parskip=half}}
\makeatother
\usepackage{xcolor}
\IfFileExists{xurl.sty}{\usepackage{xurl}}{} % add URL line breaks if available
\IfFileExists{bookmark.sty}{\usepackage{bookmark}}{\usepackage{hyperref}}
\hypersetup{
  pdftitle={Case study},
  pdfauthor={Junyuan Zhang},
  hidelinks,
  pdfcreator={LaTeX via pandoc}}
\urlstyle{same} % disable monospaced font for URLs
\usepackage[margin=1in]{geometry}
\usepackage{color}
\usepackage{fancyvrb}
\newcommand{\VerbBar}{|}
\newcommand{\VERB}{\Verb[commandchars=\\\{\}]}
\DefineVerbatimEnvironment{Highlighting}{Verbatim}{commandchars=\\\{\}}
% Add ',fontsize=\small' for more characters per line
\usepackage{framed}
\definecolor{shadecolor}{RGB}{248,248,248}
\newenvironment{Shaded}{\begin{snugshade}}{\end{snugshade}}
\newcommand{\AlertTok}[1]{\textcolor[rgb]{0.94,0.16,0.16}{#1}}
\newcommand{\AnnotationTok}[1]{\textcolor[rgb]{0.56,0.35,0.01}{\textbf{\textit{#1}}}}
\newcommand{\AttributeTok}[1]{\textcolor[rgb]{0.77,0.63,0.00}{#1}}
\newcommand{\BaseNTok}[1]{\textcolor[rgb]{0.00,0.00,0.81}{#1}}
\newcommand{\BuiltInTok}[1]{#1}
\newcommand{\CharTok}[1]{\textcolor[rgb]{0.31,0.60,0.02}{#1}}
\newcommand{\CommentTok}[1]{\textcolor[rgb]{0.56,0.35,0.01}{\textit{#1}}}
\newcommand{\CommentVarTok}[1]{\textcolor[rgb]{0.56,0.35,0.01}{\textbf{\textit{#1}}}}
\newcommand{\ConstantTok}[1]{\textcolor[rgb]{0.00,0.00,0.00}{#1}}
\newcommand{\ControlFlowTok}[1]{\textcolor[rgb]{0.13,0.29,0.53}{\textbf{#1}}}
\newcommand{\DataTypeTok}[1]{\textcolor[rgb]{0.13,0.29,0.53}{#1}}
\newcommand{\DecValTok}[1]{\textcolor[rgb]{0.00,0.00,0.81}{#1}}
\newcommand{\DocumentationTok}[1]{\textcolor[rgb]{0.56,0.35,0.01}{\textbf{\textit{#1}}}}
\newcommand{\ErrorTok}[1]{\textcolor[rgb]{0.64,0.00,0.00}{\textbf{#1}}}
\newcommand{\ExtensionTok}[1]{#1}
\newcommand{\FloatTok}[1]{\textcolor[rgb]{0.00,0.00,0.81}{#1}}
\newcommand{\FunctionTok}[1]{\textcolor[rgb]{0.00,0.00,0.00}{#1}}
\newcommand{\ImportTok}[1]{#1}
\newcommand{\InformationTok}[1]{\textcolor[rgb]{0.56,0.35,0.01}{\textbf{\textit{#1}}}}
\newcommand{\KeywordTok}[1]{\textcolor[rgb]{0.13,0.29,0.53}{\textbf{#1}}}
\newcommand{\NormalTok}[1]{#1}
\newcommand{\OperatorTok}[1]{\textcolor[rgb]{0.81,0.36,0.00}{\textbf{#1}}}
\newcommand{\OtherTok}[1]{\textcolor[rgb]{0.56,0.35,0.01}{#1}}
\newcommand{\PreprocessorTok}[1]{\textcolor[rgb]{0.56,0.35,0.01}{\textit{#1}}}
\newcommand{\RegionMarkerTok}[1]{#1}
\newcommand{\SpecialCharTok}[1]{\textcolor[rgb]{0.00,0.00,0.00}{#1}}
\newcommand{\SpecialStringTok}[1]{\textcolor[rgb]{0.31,0.60,0.02}{#1}}
\newcommand{\StringTok}[1]{\textcolor[rgb]{0.31,0.60,0.02}{#1}}
\newcommand{\VariableTok}[1]{\textcolor[rgb]{0.00,0.00,0.00}{#1}}
\newcommand{\VerbatimStringTok}[1]{\textcolor[rgb]{0.31,0.60,0.02}{#1}}
\newcommand{\WarningTok}[1]{\textcolor[rgb]{0.56,0.35,0.01}{\textbf{\textit{#1}}}}
\usepackage{graphicx,grffile}
\makeatletter
\def\maxwidth{\ifdim\Gin@nat@width>\linewidth\linewidth\else\Gin@nat@width\fi}
\def\maxheight{\ifdim\Gin@nat@height>\textheight\textheight\else\Gin@nat@height\fi}
\makeatother
% Scale images if necessary, so that they will not overflow the page
% margins by default, and it is still possible to overwrite the defaults
% using explicit options in \includegraphics[width, height, ...]{}
\setkeys{Gin}{width=\maxwidth,height=\maxheight,keepaspectratio}
% Set default figure placement to htbp
\makeatletter
\def\fps@figure{htbp}
\makeatother
\setlength{\emergencystretch}{3em} % prevent overfull lines
\providecommand{\tightlist}{%
  \setlength{\itemsep}{0pt}\setlength{\parskip}{0pt}}
\setcounter{secnumdepth}{-\maxdimen} % remove section numbering

\title{Case study}
\author{Junyuan Zhang}
\date{10/21/2021}

\begin{document}
\maketitle

\hypertarget{r-markdown}{%
\subsection{R Markdown}\label{r-markdown}}

Introduction

In this case study, we want to investigate the association between
alcohol level of ''Vinho Verde'' wine and physio-chemical information.
The predictor included in our full model are:

fixed acidity

volatile acidity

citric acid

residual sugar

chlorides

free sulfur dioxide

total sulfur dioxide

density

pH

sulphates

\begin{Shaded}
\begin{Highlighting}[]
\KeywordTok{library}\NormalTok{(ggplot2)}
\end{Highlighting}
\end{Shaded}

\begin{verbatim}
## Warning: package 'ggplot2' was built under R version 4.0.3
\end{verbatim}

\begin{Shaded}
\begin{Highlighting}[]
\KeywordTok{library}\NormalTok{(faraway)}
\end{Highlighting}
\end{Shaded}

\begin{verbatim}
## Warning: package 'faraway' was built under R version 4.0.5
\end{verbatim}

\begin{Shaded}
\begin{Highlighting}[]
\NormalTok{redwines <-}\StringTok{ }\KeywordTok{read.csv}\NormalTok{(}\StringTok{"redwines.csv"}\NormalTok{, }\DataTypeTok{header =} \OtherTok{TRUE}\NormalTok{)}
\KeywordTok{dim}\NormalTok{(redwines)}
\end{Highlighting}
\end{Shaded}

\begin{verbatim}
## [1] 1599   12
\end{verbatim}

\begin{Shaded}
\begin{Highlighting}[]
\NormalTok{redwines =}\StringTok{ }\NormalTok{redwines[, }\OperatorTok{-}\StringTok{ }\DecValTok{12}\NormalTok{] }\CommentTok{#drop the quality}
\NormalTok{redwines.full =}\StringTok{ }\KeywordTok{lm}\NormalTok{(alcohol }\OperatorTok{~}\NormalTok{.,}\DataTypeTok{data =}\NormalTok{ redwines)}
\KeywordTok{summary}\NormalTok{(redwines.full)}
\end{Highlighting}
\end{Shaded}

\begin{verbatim}
## 
## Call:
## lm(formula = alcohol ~ ., data = redwines)
## 
## Residuals:
##      Min       1Q   Median       3Q      Max 
## -2.07175 -0.39267 -0.04056  0.35396  2.44365 
## 
## Coefficients:
##                        Estimate Std. Error t value Pr(>|t|)    
## (Intercept)           6.072e+02  1.308e+01  46.419  < 2e-16 ***
## fixed.acidity         5.324e-01  2.064e-02  25.796  < 2e-16 ***
## volatile.acidity      3.608e-01  1.144e-01   3.154 0.001638 ** 
## citric.acid           8.306e-01  1.379e-01   6.024 2.11e-09 ***
## residual.sugar        2.844e-01  1.229e-02  23.135  < 2e-16 ***
## chlorides            -1.462e+00  3.956e-01  -3.696 0.000227 ***
## free.sulfur.dioxide  -2.143e-03  2.057e-03  -1.042 0.297517    
## total.sulfur.dioxide -2.296e-03  6.881e-04  -3.336 0.000868 ***
## density              -6.174e+02  1.342e+01 -45.998  < 2e-16 ***
## pH                    3.762e+00  1.551e-01  24.263  < 2e-16 ***
## sulphates             1.247e+00  1.037e-01  12.020  < 2e-16 ***
## ---
## Signif. codes:  0 '***' 0.001 '**' 0.01 '*' 0.05 '.' 0.1 ' ' 1
## 
## Residual standard error: 0.614 on 1588 degrees of freedom
## Multiple R-squared:  0.6701, Adjusted R-squared:  0.668 
## F-statistic: 322.5 on 10 and 1588 DF,  p-value: < 2.2e-16
\end{verbatim}

We first fit the full model that contains all predictors. By looking at
the Anova table, we found that the free sulfur dioxide predictor is
insignificant since its p-value is much greater than 0.05. Thus we need
consider drop this predictor

\begin{Shaded}
\begin{Highlighting}[]
\NormalTok{redwines.reduced =}\StringTok{ }\KeywordTok{lm}\NormalTok{(alcohol }\OperatorTok{~}\StringTok{ }\NormalTok{fixed.acidity }\OperatorTok{+}
\StringTok{    }\NormalTok{volatile.acidity }\OperatorTok{+}\StringTok{ }\NormalTok{citric.acid }\OperatorTok{+}\StringTok{ }\NormalTok{residual.sugar }\OperatorTok{+}\StringTok{ }
\StringTok{    }\NormalTok{chlorides }\OperatorTok{+}\StringTok{ }\NormalTok{total.sulfur.dioxide }\OperatorTok{+}\StringTok{ }
\StringTok{    }\NormalTok{density }\OperatorTok{+}\StringTok{ }\NormalTok{pH }\OperatorTok{+}\StringTok{ }\NormalTok{sulphates, }\DataTypeTok{data =}\NormalTok{ redwines) }\CommentTok{# drop free_sulfur_dioxide}

\KeywordTok{anova}\NormalTok{(redwines.reduced, redwines.full)}
\end{Highlighting}
\end{Shaded}

\begin{verbatim}
## Analysis of Variance Table
## 
## Model 1: alcohol ~ fixed.acidity + volatile.acidity + citric.acid + residual.sugar + 
##     chlorides + total.sulfur.dioxide + density + pH + sulphates
## Model 2: alcohol ~ fixed.acidity + volatile.acidity + citric.acid + residual.sugar + 
##     chlorides + free.sulfur.dioxide + total.sulfur.dioxide + 
##     density + pH + sulphates
##   Res.Df    RSS Df Sum of Sq     F Pr(>F)
## 1   1589 599.11                          
## 2   1588 598.70  1   0.40944 1.086 0.2975
\end{verbatim}

\begin{Shaded}
\begin{Highlighting}[]
\CommentTok{# don't reject null so reduced model is better}
\end{Highlighting}
\end{Shaded}

We fit a new model called redwines.reduced above without the free sulfur
dioxide. We made the hyphothesis that \(H_0 :\) reduced model is better,
\(H_a:\) full model is needed. Also, by looking the anova table above,
the \(p\) value is larger than 0.05. Thus we do not reject the null
hypothesis that reduced model is better. We confirm that can drop free
sulfur dioxide predictor. The reduced model will be used in following
analysis

\begin{Shaded}
\begin{Highlighting}[]
\CommentTok{#B outlier test}
\KeywordTok{summary}\NormalTok{(redwines.reduced)}
\end{Highlighting}
\end{Shaded}

\begin{verbatim}
## 
## Call:
## lm(formula = alcohol ~ fixed.acidity + volatile.acidity + citric.acid + 
##     residual.sugar + chlorides + total.sulfur.dioxide + density + 
##     pH + sulphates, data = redwines)
## 
## Residuals:
##      Min       1Q   Median       3Q      Max 
## -2.06145 -0.39706 -0.03917  0.34928  2.44848 
## 
## Coefficients:
##                        Estimate Std. Error t value Pr(>|t|)    
## (Intercept)           6.059e+02  1.302e+01  46.535  < 2e-16 ***
## fixed.acidity         5.300e-01  2.051e-02  25.846  < 2e-16 ***
## volatile.acidity      3.809e-01  1.128e-01   3.377 0.000749 ***
## citric.acid           8.548e-01  1.359e-01   6.289 4.12e-10 ***
## residual.sugar        2.827e-01  1.219e-02  23.198  < 2e-16 ***
## chlorides            -1.487e+00  3.949e-01  -3.766 0.000172 ***
## total.sulfur.dioxide -2.775e-03  5.123e-04  -5.416 7.02e-08 ***
## density              -6.160e+02  1.335e+01 -46.125  < 2e-16 ***
## pH                    3.739e+00  1.534e-01  24.369  < 2e-16 ***
## sulphates             1.242e+00  1.036e-01  11.984  < 2e-16 ***
## ---
## Signif. codes:  0 '***' 0.001 '**' 0.01 '*' 0.05 '.' 0.1 ' ' 1
## 
## Residual standard error: 0.614 on 1589 degrees of freedom
## Multiple R-squared:  0.6699, Adjusted R-squared:  0.668 
## F-statistic: 358.2 on 9 and 1589 DF,  p-value: < 2.2e-16
\end{verbatim}

The reduced model above now has all predictors significant

\begin{Shaded}
\begin{Highlighting}[]
\NormalTok{n =}\StringTok{ }\KeywordTok{dim}\NormalTok{(redwines)[}\DecValTok{1}\NormalTok{]}
\NormalTok{p =}\StringTok{ }\DecValTok{10} \CommentTok{# droped 1}
\NormalTok{critical =}\StringTok{ }\KeywordTok{qt}\NormalTok{(.}\DecValTok{05}\OperatorTok{/}\NormalTok{(}\DecValTok{2}\OperatorTok{*}\NormalTok{n), n}\OperatorTok{-}\NormalTok{p}\DecValTok{-1}\NormalTok{)}
\NormalTok{redwines.resid =}\StringTok{ }\KeywordTok{rstudent}\NormalTok{(redwines.reduced)}

\KeywordTok{sort}\NormalTok{(}\KeywordTok{abs}\NormalTok{(redwines.resid), }\DataTypeTok{decreasing =} \OtherTok{TRUE}\NormalTok{)[}\DecValTok{1}\OperatorTok{:}\DecValTok{10}\NormalTok{]}
\end{Highlighting}
\end{Shaded}

\begin{verbatim}
##      652      560      565      396      354      557      559      494 
## 4.038199 4.018534 4.018534 3.952545 3.858127 3.694477 3.694477 3.670237 
##      500      268 
## 3.670237 3.455404
\end{verbatim}

\begin{Shaded}
\begin{Highlighting}[]
\CommentTok{# no outliers}
\end{Highlighting}
\end{Shaded}

\begin{Shaded}
\begin{Highlighting}[]
\CommentTok{# high leverage}
\NormalTok{lev =}\StringTok{ }\KeywordTok{lm.influence}\NormalTok{(redwines.reduced)}\OperatorTok{$}\NormalTok{hat}
\KeywordTok{head}\NormalTok{(lev)}
\end{Highlighting}
\end{Shaded}

\begin{verbatim}
##           1           2           3           4           5           6 
## 0.003515642 0.005286754 0.002807848 0.004520270 0.003515642 0.003660375
\end{verbatim}

\begin{Shaded}
\begin{Highlighting}[]
\NormalTok{leverages =}\StringTok{ }\NormalTok{lev[lev}\OperatorTok{>}\NormalTok{(}\DecValTok{2}\OperatorTok{*}\NormalTok{p}\OperatorTok{/}\NormalTok{n)]}
\KeywordTok{head}\NormalTok{(leverages)}
\end{Highlighting}
\end{Shaded}

\begin{verbatim}
##         14         18         20         34         39         43 
## 0.02545301 0.02737606 0.02204734 0.02383297 0.01514653 0.02155009
\end{verbatim}

\begin{Shaded}
\begin{Highlighting}[]
\KeywordTok{length}\NormalTok{(leverages) }\OperatorTok{/}\StringTok{ }\NormalTok{n }\CommentTok{# 7 percent of points are high leverage}
\end{Highlighting}
\end{Shaded}

\begin{verbatim}
## [1] 0.07066917
\end{verbatim}

\begin{Shaded}
\begin{Highlighting}[]
\KeywordTok{halfnorm}\NormalTok{(lev, }\DecValTok{6}\NormalTok{, }\DataTypeTok{labs=}\KeywordTok{as.character}\NormalTok{(}\DecValTok{1}\OperatorTok{:}\KeywordTok{length}\NormalTok{(lev)), }\DataTypeTok{ylab =} \StringTok{"Leverages"}\NormalTok{)}
\end{Highlighting}
\end{Shaded}

\includegraphics{CS1_files/figure-latex/unnamed-chunk-4-1.pdf}

\begin{Shaded}
\begin{Highlighting}[]
\CommentTok{# constant variance}
\KeywordTok{plot}\NormalTok{(redwines.reduced, }\DataTypeTok{which =}\DecValTok{1}\NormalTok{)}
\end{Highlighting}
\end{Shaded}

\includegraphics{CS1_files/figure-latex/unnamed-chunk-5-1.pdf}

\begin{Shaded}
\begin{Highlighting}[]
\KeywordTok{library}\NormalTok{(lmtest)}
\end{Highlighting}
\end{Shaded}

\begin{verbatim}
## Warning: package 'lmtest' was built under R version 4.0.5
\end{verbatim}

\begin{verbatim}
## Loading required package: zoo
\end{verbatim}

\begin{verbatim}
## Warning: package 'zoo' was built under R version 4.0.5
\end{verbatim}

\begin{verbatim}
## 
## Attaching package: 'zoo'
\end{verbatim}

\begin{verbatim}
## The following objects are masked from 'package:base':
## 
##     as.Date, as.Date.numeric
\end{verbatim}

\begin{Shaded}
\begin{Highlighting}[]
\KeywordTok{bptest}\NormalTok{(redwines.reduced)}
\end{Highlighting}
\end{Shaded}

\begin{verbatim}
## 
##  studentized Breusch-Pagan test
## 
## data:  redwines.reduced
## BP = 169.31, df = 9, p-value < 2.2e-16
\end{verbatim}

\begin{Shaded}
\begin{Highlighting}[]
\CommentTok{# reject const variance}
\end{Highlighting}
\end{Shaded}

\begin{Shaded}
\begin{Highlighting}[]
\CommentTok{# normality}
\KeywordTok{plot}\NormalTok{(redwines.reduced, }\DataTypeTok{which =}  \DecValTok{2}\NormalTok{)}
\end{Highlighting}
\end{Shaded}

\includegraphics{CS1_files/figure-latex/unnamed-chunk-6-1.pdf}

\begin{Shaded}
\begin{Highlighting}[]
\KeywordTok{hist}\NormalTok{(redwines.reduced}\OperatorTok{$}\NormalTok{residuals)}
\end{Highlighting}
\end{Shaded}

\includegraphics{CS1_files/figure-latex/unnamed-chunk-6-2.pdf}

\begin{Shaded}
\begin{Highlighting}[]
\KeywordTok{ks.test}\NormalTok{(}\KeywordTok{residuals}\NormalTok{(redwines.reduced), }\DataTypeTok{y =}\NormalTok{ pnorm)}
\end{Highlighting}
\end{Shaded}

\begin{verbatim}
## Warning in ks.test(residuals(redwines.reduced), y = pnorm): ties should not be
## present for the Kolmogorov-Smirnov test
\end{verbatim}

\begin{verbatim}
## 
##  One-sample Kolmogorov-Smirnov test
## 
## data:  residuals(redwines.reduced)
## D = 0.14307, p-value < 2.2e-16
## alternative hypothesis: two-sided
\end{verbatim}

\begin{Shaded}
\begin{Highlighting}[]
\CommentTok{# reject normality}
\end{Highlighting}
\end{Shaded}

\begin{Shaded}
\begin{Highlighting}[]
\CommentTok{# box transformation}

\KeywordTok{library}\NormalTok{(MASS)}
\NormalTok{redwines.boxcox =}\StringTok{ }\KeywordTok{boxcox}\NormalTok{(redwines.reduced, }\DataTypeTok{lambda=}\KeywordTok{seq}\NormalTok{(}\OperatorTok{-}\DecValTok{2}\NormalTok{, }\DecValTok{2}\NormalTok{, }\DataTypeTok{length=}\DecValTok{400}\NormalTok{))}
\end{Highlighting}
\end{Shaded}

\includegraphics{CS1_files/figure-latex/unnamed-chunk-7-1.pdf}

\begin{Shaded}
\begin{Highlighting}[]
\NormalTok{lambda =}\StringTok{ }\NormalTok{redwines.boxcox}\OperatorTok{$}\NormalTok{x[redwines.boxcox}\OperatorTok{$}\NormalTok{y }\OperatorTok{==}\StringTok{ }\KeywordTok{max}\NormalTok{(redwines.boxcox}\OperatorTok{$}\NormalTok{y)]}

\NormalTok{redwines}\OperatorTok{$}\NormalTok{alcohol.new =}\StringTok{ }\NormalTok{((redwines}\OperatorTok{$}\NormalTok{alcohol)}\OperatorTok{^}\NormalTok{lambda }\OperatorTok{-}\StringTok{ }\DecValTok{1}\NormalTok{) }\OperatorTok{/}\StringTok{ }\NormalTok{lambda}

\NormalTok{redwines.transform =}\StringTok{ }\KeywordTok{lm}\NormalTok{(alcohol.new }\OperatorTok{~}\StringTok{ }\NormalTok{fixed.acidity }\OperatorTok{+}
\StringTok{    }\NormalTok{volatile.acidity }\OperatorTok{+}\StringTok{ }\NormalTok{citric.acid }\OperatorTok{+}\StringTok{ }\NormalTok{residual.sugar }\OperatorTok{+}\StringTok{ }
\StringTok{    }\NormalTok{chlorides }\OperatorTok{+}\StringTok{ }\NormalTok{total.sulfur.dioxide }\OperatorTok{+}\StringTok{ }
\StringTok{    }\NormalTok{density }\OperatorTok{+}\StringTok{ }\NormalTok{pH }\OperatorTok{+}\StringTok{ }\NormalTok{sulphates, }\DataTypeTok{data =}\NormalTok{ redwines)}

\KeywordTok{summary}\NormalTok{(redwines.transform)}
\end{Highlighting}
\end{Shaded}

\begin{verbatim}
## 
## Call:
## lm(formula = alcohol.new ~ fixed.acidity + volatile.acidity + 
##     citric.acid + residual.sugar + chlorides + total.sulfur.dioxide + 
##     density + pH + sulphates, data = redwines)
## 
## Residuals:
##        Min         1Q     Median         3Q        Max 
## -0.0056710 -0.0010157 -0.0000064  0.0010000  0.0064187 
## 
## Coefficients:
##                        Estimate Std. Error t value Pr(>|t|)    
## (Intercept)           2.080e+00  3.349e-02  62.118  < 2e-16 ***
## fixed.acidity         1.307e-03  5.274e-05  24.786  < 2e-16 ***
## volatile.acidity      9.860e-04  2.901e-04   3.399 0.000692 ***
## citric.acid           2.066e-03  3.496e-04   5.908 4.22e-09 ***
## residual.sugar        6.819e-04  3.135e-05  21.752  < 2e-16 ***
## chlorides            -4.683e-03  1.016e-03  -4.610 4.34e-06 ***
## total.sulfur.dioxide -8.419e-06  1.318e-06  -6.389 2.19e-10 ***
## density              -1.495e+00  3.435e-02 -43.531  < 2e-16 ***
## pH                    9.259e-03  3.946e-04  23.461  < 2e-16 ***
## sulphates             3.190e-03  2.665e-04  11.970  < 2e-16 ***
## ---
## Signif. codes:  0 '***' 0.001 '**' 0.01 '*' 0.05 '.' 0.1 ' ' 1
## 
## Residual standard error: 0.001579 on 1589 degrees of freedom
## Multiple R-squared:  0.6509, Adjusted R-squared:  0.649 
## F-statistic: 329.3 on 9 and 1589 DF,  p-value: < 2.2e-16
\end{verbatim}

\begin{Shaded}
\begin{Highlighting}[]
\KeywordTok{plot}\NormalTok{(redwines.transform, }\DataTypeTok{which =} \DecValTok{1}\NormalTok{)}
\end{Highlighting}
\end{Shaded}

\includegraphics{CS1_files/figure-latex/unnamed-chunk-8-1.pdf}

\begin{Shaded}
\begin{Highlighting}[]
\KeywordTok{bptest}\NormalTok{(redwines.transform)}
\end{Highlighting}
\end{Shaded}

\begin{verbatim}
## 
##  studentized Breusch-Pagan test
## 
## data:  redwines.transform
## BP = 184.17, df = 9, p-value < 2.2e-16
\end{verbatim}

\begin{Shaded}
\begin{Highlighting}[]
\KeywordTok{hist}\NormalTok{(redwines.transform}\OperatorTok{$}\NormalTok{residuals)}
\end{Highlighting}
\end{Shaded}

\includegraphics{CS1_files/figure-latex/unnamed-chunk-8-2.pdf}

\begin{Shaded}
\begin{Highlighting}[]
\KeywordTok{ks.test}\NormalTok{(}\KeywordTok{residuals}\NormalTok{(redwines.transform), }\DataTypeTok{y=}\NormalTok{pnorm)}
\end{Highlighting}
\end{Shaded}

\begin{verbatim}
## Warning in ks.test(residuals(redwines.transform), y = pnorm): ties should not be
## present for the Kolmogorov-Smirnov test
\end{verbatim}

\begin{verbatim}
## 
##  One-sample Kolmogorov-Smirnov test
## 
## data:  residuals(redwines.transform)
## D = 0.49774, p-value < 2.2e-16
## alternative hypothesis: two-sided
\end{verbatim}

\begin{Shaded}
\begin{Highlighting}[]
\KeywordTok{plot}\NormalTok{(redwines.transform, }\DataTypeTok{which=}\DecValTok{2}\NormalTok{)}
\end{Highlighting}
\end{Shaded}

\includegraphics{CS1_files/figure-latex/unnamed-chunk-9-1.pdf}

\end{document}
